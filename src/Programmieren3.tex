\documentclass{article}

\topmargin = 1pt
\headheight = 20pt
\headsep = 10pt
\oddsidemargin = 1pt
\marginparwidth = 1pt
\textwidth = 455pt
\textheight = 600pt

\usepackage{makeidx}
\usepackage{minted}
\usepackage{tikz}
\usepackage{fancyhdr}
\usepackage{hyperref} % must come last

\pagestyle{fancy}

% \usemintedstyle{tango}

% \definecolor{bg}{rgb}{0.95, 0.95, 0.95}
\definecolor{bg}{rgb}{1, 1, 1}

\usetikzlibrary{arrows}

\newcommand{\displayCode}[1] {
  \inputminted[linenos,
               bgcolor=bg,
               frame=lines,
               fontfamily=courier,
               fontsize=\small]
               {c++}
               {../../#1}
}

\newcommand{\displayCodeFromTo}[3] {
  \inputminted[linenos,
               bgcolor=bg,
               frame=lines,
               fontfamily=courier,
               fontsize=\small,
               firstnumber=#2,
               firstline=#2,
               lastline=#3]
               {c++}
               {../../#1}
}

\newcommand{\includeChapter}[1] {
  \newpage
  \input ../src/#1.tex
}

\newcommand{\paragraphWithLineBreak}[1] {
  \paragraph{#1} ~\\
}


\begin{document}

  \title{Programmieren 3 Zusammenfassung}
  \date{11. Januar 2014}
  \author{Samuel Mueller}
  \maketitle

  \newpage

  \tableofcontents
  \newpage

  \includeChapter{terminology}

  \includeChapter{basic_concepts}

  \includeChapter{classes}

  \includeChapter{immutability}

  \includeChapter{streams}

  \includeChapter{iterators}

  \includeChapter{containers}

  \includeChapter{algorithms}

  \includeChapter{bind}

  \includeChapter{templates}

  \includeChapter{references_and_pointers}

  \includeChapter{compile_time_calculation}

  \includeChapter{good_to_know}

  \includeChapter{appendix}

\end{document}